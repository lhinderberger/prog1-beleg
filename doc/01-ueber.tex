\section{Über dieses Dokument}

Dieses Dokument stellt die begleitende Dokumentation meiner Belegarbeit im Modul Programmierung 1
im Wintersemester 2016/17 an der HTW Dresden dar.

Es soll die entscheidenden Schritte im Verlauf der Entwicklung des Beleges dokumentieren und
nachvollziehbar machen.

Der hauptsächliche Teil der Belegarbeit ergibt sich aus dem beiliegenden Quellcode sowie der daraus resultierenden
ausführbaren Anwendung (Materialdatenbank).

\subsection{Vorgehensmodell}
TODO QUELLE: Lichter Software Engineering!!!

Die Entwicklung erfolgte nach dem Wasserfall-Modell (ausgenommen der Wartungsphase, da die abgelieferte Software
nur ein akademisches Beispiel darstellt und nicht praktisch eingesetzt wird), leicht modifiziert durch den Vorzug der
Unit-Tests in / vor die Implementierungsphase (Test Driven Development). Dementsprechend ist auch dieses Dokument
in die Abschnitte ``Analyse'', ``Entwurf'', ``Implementierung'', ``Test'' und ``Inbetriebnahme''
unterteilt, um die einzelnen Stationen entlang des Entwicklungsprozesses geordnet abzubilden.

Das Wasserfall-Modell erschien -- trotz, dass es in der Praxis als veraltet angesehen wird -- als geeigneter
Entwicklungsprozess, da an der Software weder inkrementell Verbesserungen vorgenommen noch zukünftige etwaige
``Kundenwünsche'' erfüllt werden müssen. Die Aufgabenstellung (vgl. \ref{aufgabenstellung}) ist vielmehr klar und
unveränderlich, die Abnahme des Belegs erfolgt einmalig und nicht im Rahmen von ``Feedback-Schleifen'', wie sie in
der iterativen / agilen Software-Entwicklung üblich sind.

Zwar ist die Notwendigkeit des strukturierten Vorgehens nach einem Prozessmodell (selbst, wenn es ein ``Minimalmodell''
wie das Wasserfallmodell ist) und die ausführliche Dokumentation für einen Grundlagenbeleg durchaus zu hinterfragen,
jedoch erachte ich beides für sehr hilfreich und -- da in der Praxis bewährt und lieb gewonnen -- auch ein Stück weit für
unverzichtbar. Weiterhin bin ich der subjektiven Überzeugung, dass nur so ein gewisser Qualitätsstandard gehalten werden
kann, den ich aus der Praxis (überwiegend) gewohnt bin und dem ich mich auch im Studium verpflichtet fühle.
Die Erfahrungen aus der Anfertigung der Belegarbeit haben dies erneut bestätigt -- so wären beispielsweise viele, teils
hoch kritische, Fehler betrefflich der Datenintegrität nicht, oder erst spät aufgefallen, hätte ich z.B. nicht nach dem
Prinzip des Test Driven Development gearbeitet. Auch der ``Rote Faden'', den Spezifikation und Grobentwurf bieten,
hat sich während der Implementierung immer wieder als höchst hilfreich erwiesen; der hierfür nötige Mehraufwand hat sich
meines subjektiven Erachtens nach gelohnt.
