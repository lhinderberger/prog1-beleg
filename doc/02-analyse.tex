\section{Analyse}
\subsection{Aufgabenstellung und direkte Anforderungen}
Zu bearbeiten war Aufgabe Nr. 2, unter Verwendung von Methode Nr. 1, errechnet aus der Matrikelnummer.
Die Aufgabenstellung des Belegs besagt in diesem Fall, dass eine Materialverwaltung unter Verwendung der GTK-Bibliothek
für die Benutzeroberfläche programmiert werden soll.

\bigskip

Aus der online verfügbaren, vollständigen Aufgabenstellung\footnote{\url{http://www.informatik.htw-dresden.de/~beck/PSPI/Belegaufgaben/},
abgerufen am 8.12.2016} können folgende direkte Anforderungen entnommen werden:

\begin{itemize}
\item Das Programm soll Datensätze verwalten -- also anlegen, sortiert tabellarisch auflisten, suchen, bearbeiten,
löschen sowie einzelne Datensätze anzeigen -- können
\item Ein Datensatz besteht hierbei aus Artikelbezeichnung, Artikelnummer und Lagerbestand
\item Die im Programm erfassten Daten sollen im Dateisystem gespeichert (bzw. davon wieder ausgelesen) werden können.
Hierbei ist nach klärendem Gespräch auch die Benutzung des Datenbanksystems SQLite erlaubt.
\item Der Lagerbestand soll explizit über einen gesonderten Menüpunkt veränderbar sein
\item Beim Verlassen der Anwendung sollen Daten, die sich während des Programmablaufs geändert werden
gespeichert werden können.\footnote{Hierbei wird dem Anwender eine Wahl gelassen werden, ob er wirklich speichern, oder
das Programm ohne Berücksichtigung der vorgenommenen Änderungen verlassen möchte}
\item Für die Umsetzung der Benutzeroberfläche muss zwingend die Bibliothek GTK+ verwendet werden
\item Es wird ein modularer Aufbau vorausgesetzt - Das Programm ist in mehrere sinnvolle, einzeln kompilierbare Module
zu unterteilen\footnote{Dies schließt m.E.n. nicht die Nutzung gemeinsamer Typdefinitionen und Schnittstellen aus}
Hierbei sind mindestens drei Module anzufertigen
\item Quelltexte sind sorgsam zu dokumentieren -- insbesondere ist die Urheberschaft im Programmkopf zu
kennzeichnen\footnote{Bei der Quellcodedokumentation richte ich mich nach \cite[Kap.4]{Martin:CleanCode}}
\item Das Programm muss auf den Laborrechnern der Fakultät Informatik an der HTW Dresden übersetz- und ausführbar sein.
\item Teil der Aufgabenstellung ist die verpflichtende Verwendung dynamischer Speicherverwaltung per malloc/free\footnote{
Ich werde den Einsatz von malloc/free dennoch auf das Notwendigste begrenzen, da gerade unter C die dynamische
Speicherverwaltung erfahrungsgemäß eine ergiebige Fehlerquelle darstellt.}
\end{itemize}

\subsection{Ergänzende Anforderungen}
Das oben genannte Minimum möchte ich außerdem noch um folgende Anforderungen ergänzen, die sich bei näherer Betrachtung
der Aufgabenstellung als sinnvoll und einfach umzusetzen bzw. allgemein als bewährte Verfahrensweise herausgestellt haben.

\begin{itemize}
\item Die Benutzeroberfläche soll selbsterklärend und fehlertolerant sein und möglichst wenig (also ideal überhaupt keine)
Nutzerdokumentation erfordern
\item Das Programm -- insbesondere die Module für Abstraktion, Speicherung und Verwaltung von Datensätzen (vgl.
Kernschicht im Grobentwurf) -- soll möglichst umfassend durch automatische Softwaretests abgedeckt werden.
Hierbei soll nach Möglichkeit auch das Vorgehen der Testgetriebenen Entwicklung (TDD) erprobt werden.
\item Dem Nutzer soll ermöglicht werden, optional zu einem Datensatz auch ein Bild zu speichern
\item Die in GTK+ vorhandenen Schnittstellen zur Internationalisierung sollen genutzt werden.\footnote{Selbst wenn das
Programm praktisch gesehen wohl einsprachig bleiben wird, ist dies noch immer der für GTK+-Anwendungen übliche Weg
und mit vernachlässigbar wenig Mehraufwand umsetzbar.}
\item Änderungen an den Datensätzen sollen im Arbeitsspeicher gehalten und erst mit Senden des Speicherbefehls in das
Dateisystem geschrieben werden. Wird kein Speicherbefehl gegeben (auch nicht auf Nachfrage beim Verlassen der Anwendung),
so haben diese Änderungen zu verfallen.
\item Das Speichern der Datenbankdatei soll auch unter neuem Dateinamen erfolgen können.
\item Der aktuell angezeigte Datensatz bzw. die aktuell angezeigte Auflistung sollen, wenn möglich, in ein CSV-basiertes
Format exportierbar sein.
\end{itemize}

\subsection{Domänenmodell}