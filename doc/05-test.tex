\section{Test}

\subsection{Testlauf Anwendungsfälle}
Während des abschließenden Testlaufs durch alle Anwendungsfälle sind mir an vielen Stellen im Frontend noch Dinge
aufgefallen, die nicht wie erwartet oder nicht für den Nutzer intuitiv funktioniert haben.

Beispielsweise ließ sich GTK via Glade nicht dazu überreden, das Dialogfeld zur Auswahl des Artikelbilds als modalen
Dialog darzustellen. Erst durch eine ``Notlösung'', welche sich in das \lstinline{show}-Signal des Dialogfeldes
einhakt, konnte das Problem behoben werden.

Funktioniert haben, jedoch nicht intuitiv waren hingegen zum Beispiel die Dialogfelder zum Öffnen / Anlegen einer
Datenbank. Diese haben nicht auf einen Doppelklick auf die auszuwählende Datei reagiert. Durch eine Anbindung an das
\lstinline{file-activate}-Signal der Dialoge konnte dies behoben werden.

Nach Auffindung aller nicht optimal oder fehlerhaft ausgeführten UI-Komponenten wurde schließlich versucht, das
Programm durch beabsichtigte Falscheingaben zum Absturz oder zu undefiniertem Verhalten zu bringen. Hierbei wurden
zusätzlich noch einige Fehler gefunden und abgestellt.

\subsection{Abdeckung Spezifikation}
Zuletzt habe ich noch einmal geprüft, ob ich auch wirklich alle Anforderungen umgesetzt habe. Und tatsächlich ist mir
hierbei aufgefallen, dass ich die Funktionalität zur Löschung von Artikeln vergessen hatte.
Hier hat sich wieder einmal die systematische Vorgehensweise und die ausführliche Dokumentation bezahlt gemacht.

Was die Abdeckung der Spezifikation angeht, sind nach meinem Eindruck nach dem Test alle Pflicht- und auch die
darüber hinausgehenden Anforderungen ausreichend erfüllt, sodass der Beleg nach einer probeweisen Inbetriebnahme in
der Räumen der HTW Dresden abgabereif ist. Ein endgültiges Urteil hierüber steht mir jedoch selbstverständlich nicht zu.
